% standardmäßig im 4:3 Format
% verwende \documentclass[169]{include/sikslides} %für 16:9
\documentclass{include/sikslides}

\title[Kurztitel]{Titel der Seminararbeit}
\subtitle{Seminar Embedded Systems} % Bachelor
%\subtitle{Seminar Safety-Critical Systems} % Master
\author{Vorname Nachname}
\date[\today]{\today}

\begin{document}

\titleframe

% erneutes Bauen notwendig, um Gliederung zu aktualisieren
\begin{frame}
    \frametitle{Gliederung}
    \tableofcontents[hideallsubsections]
\end{frame}

\sectionframe{Erster Teil}

\begin{frame}
  \frametitle{Überschrift}
  \begin{itemize}
  \item Aufzählungspunkt
    \begin{itemize}
    \item Eine Ebene tiefer
    \item ...
    \end{itemize}
  \item Noch ein Punkt
  \end{itemize}
\end{frame}

\sectionframe{Zweiter Teil}

\begin{frame}
    \frametitle{Überschrift}
    Text ohne Aufzählung
    \vfill
    \begin{center}
        \Large\textbf{Formatierung wie in normalen \LaTeX-Dokumenten möglich}
    \end{center}
\end{frame}

\end{document}

